\chapter*{Úvod}
\addcontentsline{toc}{chapter}{Úvod}

Městská hromadná doprava v Praze a Středočeském kraji je jeden z hlavním pilířů přepravy osob v této oblasti. Svým rozsahem a důležitostí se přímo dotýká každého z nás a její fungování do značné míry ovlivnˇuje naše konání v krátkém i dlouhém časovém horizontu.

\bigbreak

Každého cestujícího v přepravě jistě někdy trápilo zpoždění svého spoje. To člověka přivádí k myšlenkám jestli by nebylo možné určit s jakou pravidelností, pokud s nějakou, takové zpoždění vznikají.

\bigbreak

Ve vymezené oblasti operuje spousta soukromých i městských přepravců. Ti kteří spadají do naší zájmové oblasti zastřešuje organizace \gls{ropid}, která objednává jednotlivé spoje. Pro tuto práci je však důležité, že tato organizace zadala jednotlivým dopravcům vysílat aktuálním polohy jejich vozů. Tato data jsou přes zprostředkovatele zveřejnˇována na pražské datové platformě zvané Golemio, jež je ve správě společností Operátor ICT, a. s., která je vlastněná hlavním městem Praha.

\bigbreak

Nicméně v době návrhu práce kvůli právním komplikacím nebyly k dispozici real-time data z majoritního přepravce na území Prahy Dopravní podnik hl. m. Prahy. Vzhledem k povaze této práce avšak tyto data nenabývají takové důležitosti jako data od přepravců operujících mimo Prahu. Vzhledem k tomu, že zbylí dopravci využívají převážně autobusy k přepravě cestujících, jinými způsoby dopravy se tedy zabívat nebudeme.

\bigbreak

Řešení ovšem není pouze založeno na real-tim datach, ale využívá také statická data o jízdních řádech a zastávkách hromadné dopravy.

\section*{Uživatelská aplikace}

Při dispozici dat o aktuálních polohách vozidel \gls{mhd} se nabízí jejich využití tak, že budou vynesena do mapy a tím vznikne vizuálně přívětivé uživatelské prostředí pro prohlížení aktuálního stavu sítě vozidel.

\bigbreak

TODO popis aplikace

TODO testovani

TODO design




Podle získaných dat se v pracovní den vypravý přibližně (TODO spocitat pocet autobusu denne) autobusových spojů.
