\chapter*{Úvod}
\addcontentsline{toc}{chapter}{Úvod}

Městská hromadná doprava v Praze a Středočeském kraji je jedním z hlavních pilířů přepravy osob na tomto území. Jejím rozsahem a důležitostí se přímo dotýká každého z nás a její fungování do značné míry ovlivňuje naše konání v krátkém i dlouhém časovém horizontu.

\bigbreak

Každého cestujícího v přepravě jistě někdy trápilo zpoždění svého spoje. To člověka přivádí k myšlenkám, zda by nebylo možné určit s jakou pravidelností, pokud s nějakou, takové zpoždění vznikají. A zda by nemohl být včas informován o vzniklé anomálii a vzniklém zpoždění.

\bigbreak

Ve vymezené oblasti operuje spousta soukromých i městských dopravců. Ti, kteří spadají do naší zájmové oblasti, zastřešují organizace \gls{ropid} a \gls{idsk}, které objednávají jednotlivé spoje. Pro naši práci je důležité, že tyto organizace zadaly jednotlivým dopravcům vysílat aktuální polohy jejich vozů. Tato data o polohách jsou přes zprostředkovatele zveřejňována na pražské datové platformě zvané Golemio\footnote{www.golemio.cz}, jež je ve správě společností Operátor ICT, a. s., která je ve vlastnictví hlavního města Praha. Takových spojů, o kterých máme všechna požadovaná data, je v pracovní den vypraveno necelých deset tisíc.\footnote{ze dne 20. 2. 2020 podle testovacích dat}

\section*{Definice problému}

Cílem této práce je zpřesnit odhad zpoždění vozidel veřejné hromadné dopravy (dále jen \gls{vhd}), zejména autobusů, na trase mezi dvěma sousedícími zastávkami. Dále pak tyto výsledky vizualizovat v mapových podkladech.


\bigbreak

V době návrhu práce, kvůli právním komplikacím a složitosti informačního systému \citet{Gurycova19}, nebyly k dispozici real-time data od majoritního dopravce na území Prahy \gls{dpp}. Jelikož je práce zaměřena na odhad zpoždění spoje mezi dvěma sousedícími zastávkami na trase, má tedy větší význam odhadovat zpoždění mezi zastávkami, mezi kterými je větší vzdálenost. A to jsou převážně spoje jednoucí mimo Prahu. Proto tato data z \gls{dpp} nenabývají takové důležitosti, jako data od dopravců operujících mimo Prahu. Vzhledem k tomu, že zbylí dopravci využívají převážně autobusy k přepravě cestujících, bude práce vypracována pouze s ohledem na autobusovou dopravu.

\bigbreak

V práci se tedy pokusíme využít dostupná otevřená real-time data k získání informací o zpoždění spojů na trase a využít je k lepším odhadům zpoždění. Řešení ovšem není založeno pouze na real-time datach, ale využívá také statická data o jízdních řádech nebo zastávkách hromadné dopravy, jejichž zdrojem je přímo \gls{ropid}\footnote{pid.cz/o-systemu/opendata/} a také mapové podklady. Ty jsou potřeba zejména pro vizualizaci zastávek a jízdních řádů, nebo vykreslení trasy spoje přímo do mapy. Avšak i tyto statická data jsou dostupná přímo z datové platformy pomocí stejného rozhraní jako data real-timová.

\bigbreak

Protože disponujeme daty o aktuálních polohách vozidel \gls{vhd}, která navíc rozšíříme o lepší odhady zpoždění, nabízí se jejich využití tím, že budou zanesena do mapy, čímž vznikne vizuálně přívětivé uživatelské prostředí pro prohlížení aktuálního stavu sítě vozidel. V práci tedy navrhujeme a implementujeme uživatelskou aplikaci, která vozidla zobrazí a bude komunikovat s uživatelem tak, že na jeho žádost zobrazí více infomací o daném spoji nebo vybrané zastávce.

\section*{Souhrn}

Klientská aplikace bude vytvořená pro webovou platformu, kód bude napsán ve standardních jazycích pro vývoj webových aplikací. Jsou jimi zejména \gls{html} a \gls{js}.

\bigbreak

Celá serverová část je napsána v jazyce Python3. Server bude sloužit pro stahování a ukládání dat do \gls{sql} databáze, konkrétně implementace MySQL. Další část back-endu bude zpřístupňovat data z databáze, určené pro klienstkou aplikaci.

\bigbreak

Jádro celé práce tvoří modul odhadu zpoždění. Zde používáme knihovny jakyka Python3 pro strojové učení scikit-learn a pro práci s velkými objemy dat NumPy. Zpoždění spojů budeme odhadovat pomocí pravděpodobnostních modelů natrénovaných na historických datech.

\bigbreak

V průběhu celé práce se setkávame s grafy a statistikami. Ty vyplývají z dat, dle kterých byla celá práce testována. Konkrétně byla data sbírána ve dnech 20. 2. 2020 až 23. 2. 2020. Dále v textu, kde se takové statistické údaje vyskytují, je vždy uvedeno i z jaké podmnožiny všech testovacích dat vycházejí. Všechny grafy, obrázky a statistické údaje uvedené v této práci, u kterých není explicitně uveden zdroj, byly vytvořeny jako součást této práce a jsou založeny na datech a poznatcích získaných v rámci této práce. Vstupní data jsou analyzována v kapitole Analýza.

\section*{Struktura dokumentu}

Tento dekument je rozdělen do 4 hlavních kapitol.

\subsection*{Analýza}

V této kapitole se zaměříme na detailní definici a popis řešeného problému. Analyzujeme vstupní data a jejich zdroj. Dále si ukážeme příklady již funkčních nástrojů vizualizující polohy vozidel a jiná data.

\subsection*{Návrh řešení}

V této kapitole definujeme funkční požadavky na dílo. Navrhneme databázi, do které budeme ukládat data, včetně algoritmu plnícího databázi. Poté navrhneme algoritmus, kterým budeme odhadovat zpoždění. A na závěr kapitoly definujeme front-endovou aplikaci včetně serverové části.

\subsection*{Implementace}

Zde popíšeme implementaci konkrétních softwarových částí práce. Tedy zpracování dat, algortimu odhadu zpoždění a klienstké aplikace.

\subsection*{Testování a evaluace}

Ukážeme, že aplikace je řádně otestována po všech stránkách a především porovnáme řešení této práce se stávajícím řešením todo. Na závěr si ukážeme zajímavé statistiky o jízdách vozidel \gls{vhd}.

\section*{Definice pojmů}

Tato práce zabíhá do velkého detailu reálií \gls{vhd}. Protože je z hlediska softwarového vývoje velmi důležité ujasnit si terminologii a předcházet tak různým nedorozuměním, definujme si pojmy, které mohou běžně splývat.

\begin{itemize}

	\item Zastávka / Stanice / Nástupiště -- Nástupiště je nejmenší bod, který jsme schopni rozlišit. Od něj odjíždějí konkrétní vozidla. Rozdíl mezi zastávkou a stanicí není jasně definován. Obecně se chápe, že stanice je většího rozsahu, tedy že má více nástupišť, nebo je obsluhována více druhy dopravních prostředků. Stanicím a zastávkám pak přísluší i pojmenování. V této práci na úrovni softwaru budeme výhradně pracovat s nástupišti. Avšak pro zlepšení čitelnosti textu používáme slovo zastávka i ve smyslu nástupiště.

	\item Jízdní řád -- Je definován posloupností zastávek, kterým náleží časy příjezdu a odjezdu. Čas projetí zastávky je počítán vždy od půlnoci, tedy není vázán na konkrétní den.

	\item Spoj -- Spoj je definován jízdním řádem, dále mu přísluší informace o jeho konečné stanici a číslo. Z definice jízdního řádu vyplývá, že jeden spoj může jet v libovolný počet dnů, ale vždy ve stejný čas.

	\item Jízda -- Jízdou myslíme jednu konkrétní realizaci spoje.

\end{itemize}.
