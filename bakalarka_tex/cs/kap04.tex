%%% Fiktivní kapitola s instrukcemi k PDF/A

\chapter{Vizualizace dat}

Pro vizualizaci dat do webového rozhraní na mapový podklad je využito prostředí Mapbox. K tomu je dále potřeba vytvořit serverovou aplikace pro komunikaci s webovou stránkou.

\subsection{Klientská část}

Tato webová strának je napsána ve standardním \gls{html} s tím, že pro stylování objektů je použit jazyk \gls{css}. Hlavní vlastnosti stránky, jako je zobrazení entit do mapy je použitý jazyk \gls{js}, zejména pak jeho možností pro zachzázení s \gls{dom} elementy. Pro připojení a načítání dat ze serveru se používá technologie \gls{ajax}ových dotazů.

\bigbreak

Koncepce klientské aplikace je taková, že žádná data nezpracovává ani nepřepočítává a zobrazuje jen data taková, která obdržela od serverové strany typicky ve formátu \gls{geojson}.

\subsubsection{Mapbox API}

Prostředí Mapbox je široce využívaný multiplatformový nástroj pro zobrazení mapového podkladu a umožnˇuje do něj zanést širokou škálu různých geometrických útvarů. Tak že mapové prostředí intuitivně interaguje s uživatelem a vývojáři mohou využití jednoduchécho \gls{api} pro zobrazení žádoucích dat do mapy.

\bigbreak

Webová aplikace této práce využívá naprosto základní funkcionality, které mapbox umožnˇuje. Jejich popis včetně načtení do webové stránky za předpokladu, že jsou splněny základní parametry webové stránky je následující.

\bigbreak

Rozhraní se do webové stránky importuje pomocí:
\begin{code}[frame=none]
<script src='https://api.tiles.mapbox.com/
	mapbox-gl-js/v1.4.0/mapbox-gl.js'></script>
<link href='https://api.tiles.mapbox.com/
	mapbox-gl-js/v1.4.0/mapbox-gl.css' rel='stylesheet' />
\end{code}

\bigbreak

Dále je potřeba vytvořit element s identifikátorem webové stránky, kde bude mapa zobrazena.

\bigbreak

Po naiportovéní je v JavaScriptu k dispozici knihovna jménem \verb-mapboxgl- pomocí, které se ovládá celé mapové prostředí. Pomocí ní je tedˇ možné vytvořit mapu.
\begin{code}[frame=none]
var map = new mapboxgl.Map({
	container: 'map', // identifikátor HTML elementu
	style: 'mapbox://styles/mapbox/streets-v11',
	center: [14.42, 50.08], // střed mapy při
		inicializaci [lng, lat]
	zoom: 10 // zoom při inicializaci
});
\end{code}

\bigbreak

Nyní stačí jen vytvořit \gls{html} element za pomocí \gls{js} a po té může být přidám do mapy funkcí:
\begin{code}[frame=none]
new mapboxgl.Marker(element)
	.setLngLat([Lng, Lat]) // zeměpisná výška a šířka
		umítění elementu
	.addTo(map);
\end{code}

\bigbreak

Pro vykreslení složitějších objektů, jako je třeba lomená čára se využívá funkce \verb-addLayer-.
\begin{code}[frame=none]
map.addLayer({
	"id": id, // identifikátor vrstvy
	"type": "line", // geometrický útvar k zobrazení
	"source": {
		"type": "geojson", // formát zdrojových dat
		"data": data // zdroj dat
	},
	"paint": {
		"line-color": "#BF93E4", // barva
		"line-width": 5 // šířka
	}
});
\end{code}

\bigbreak

K manipulaci s objekty typu \verb-Layer- se používá
\begin{code}[frame=none]
map.getLayer(id);
map.gremoveLayer(id);
\end{code}

\subsubsection{Funkce a design aplikace}

% TODO vypsat co vsechno to umi


\subsection{Serverová část}

Příchozí požadavky od klineta jsou řešeny skriptem na serverové straně. Který je napojený na databázi a z ní extrahuje potřebná data.

\bigbreak

Data jsou posílána v textové podobě ve formátu \gls{geojson}, který skrip konstruuje z dat získaných z databáze.

\bigbreak

Server reaguje na 4 typy požadavků:

\begin{itemize}
	\item \verb-get_vehicle_positions- vrátí aktuální polohy všech vozidel,

	\item \verb-get_tail- vrátí lomenou čáru popisující pohyb vozidla v uplynulých $n$ minutách podle id tripu,

	\item \verb-get_shape- vrátí lomenou čáru popisující trasu spoje podle id spoje,

	\item \verb-get_stops- vrátí seznam zastávek pro spoj podle jeho id.
\end{itemize}

\bigbreak

Server je naprogramovám pomocí Pythoní knihovny \verb-simple_server-, která slouží pouze k debugování, jak se píše v její dokumentaci. (TODO odkaz na dokumentaci). Protože se nepočítá s reálným nasezením této aplikace, není potřeba programovat robustní server. Pro takové užití by bylo třeba přejít na více vláknové řešení.























%f
