%%% Fiktivní kapitola s ukázkami sazby

\chapter{Analýza problému a jeho řešení}

V této kapitole je detailně popsán problém a způsoby jeho navrženého a současného řešení.

\section{Popis problému}

Spoje které zajištˇují hromadnou dopravu se řídí jízdními řády, které určují jejich trasu a udávají časy příjezdu a odjezdu do daných zastávek. Toto jsou zpravidla jediné refenční body u kterých jsme schopní zjistit skutečné zpoždění, nebo předjetí (dále uvažováno jako zpoždění se zápornou hodnotou).

\bigbreak

Vzhledem k tom, že délka trasy mezi dvěma refernčními body nezříka dosahuje i několika desítek kilometrů (TODO spocitat prumer a median), kde mohou vznikat mimořáné události, ale zpravidla je ovlivněna obvyklým provozem, je potřeba navrhnout systém na odhat zpoždění v půběhu jízdy.

\bigbreak

Toto celé je potřeba počítat v reálném čase, tak aby uživatelé byli dobře informování o stavu jejich spoje. Proto je potřeba zpracovávat data okamžitě po jejich vydání, spočítat odhad zpoždění a vystavit tyto data veřejně. Vzhledem k tomu, že tyto data velmi rychle zastarávají je nutné provést tento proces co možná nejrychleji.

\bigbreak

Pro vyjasnění je potřeba uvést, že se systém nesnaží předpovědět zpoždění, které spoj může nabrat vzhledem k dosavadnímu průběhy trasy. Ale snaží se odhadnou zpoždění v danný bod na trase vzhledem k obvyklému profilu jízdy.

TODO obrazek nelinearni trasy

\section{Současné řešení}

Takový algoritmus na odhat aktuálního zpoždění již exituje a je zakomponován v systému, ze kterého se čerpají data pro tuto práci. (Detailní popis dat uveden v kapitole ~\ref{cahpter:TODO later}.) Nicméně nezohlednˇuje základní parametry průběhu trasy. Tento algoritmus nahlíží na postup vozidla na trase jako na lineární funkci vůči času. Jak ale z praxe víme (TODO doplnit zdroj), rychlost vozidel není konstantní, neboli doba jízdy není linárně závislá na ujeté vzdálenosti.

\section{Analýza požadavků na uživatelskou aplikaci}

Součástí práce je i vizualizace spočítaných dat. Jinými slovy nástroj umožnˇující přístup uživatelů ke spočítaným datům.

\subsection{Poskytovatelé mapových podkladů}

K takovému účelu nejlépe poslouží vykreslení aktuálních poloh vozidel do mapy, kde se po vyžádání uživatelem tyto data zobrazí.

\bigbreak

Za účelem vytvoření dostatečně přívetivé uživatelské aplikace je nezbitné využít některého z poskytovatelů mapového \gls{api}, neboli využít již existující mapový podklad a zanést do něj získané informace.

\bigbreak

Jedním z těchto poskytovatelů je společnost Google, která má propracované mapové podklady a prostřednictvím služby Google Maps poskytuje pro tuto práci požadovanou službu. Další platformou je Mapbox, který poskytuje velmi podobné služby jako Google Maps. Nicméně natozdíl od Googlu využívá jako mapový podklad \gls{osm} {otevřená geografické data}. Protože smyslem práce je v co největší míře využít otevřená data je žádoucí využít právě Mapbox.

\bigbreak

TODO dokumentace mapbox, zeptat se jestli je to vubec nutne rozebirat

\subsubsection{Funkční požadavky}

\begin{itemize}
	\item Aplikace vykreslí interaktivní mapu Prahy a širšího okolí, kterou bude možné posunovat či oddalovat. V této mapě budou zobrazeny jednotlivé vozidla na aktuálních pozicích a budou se automaticky posouvat po mapě.

	\item Po kliknutí na vozidlo se zobrazí jeho celá trasa včetně zastávek a jeho dopočítané zpoždění.

	\item Celá aplikace bude postavena na principu server -- client. Tedy serverová strana se postará o přístup k otevřeným datům o vozidlech a jejich uložení a také obsluhu požadavků klienta. Klientská část bude webová stránka poskytující služby popsané výše. Měla by být schopná zobrazit řádově tisíce vozidel.
\end{itemize}

\subsubsection{Nefunkční požadavky}

\begin{itemize}
	\item Serverová část bude napsaná v jazyce Python 3.

	\item Webová část bude napsaná pomocí jazyků pro webové technologie, převážně v JavaScriptu.

	\item Pro vykleslení mapy bude využita služba Mapbox.

	\item Ukládání dat na serverové straně bude řešeno MySQL databází.

	\item Pro různé odhady zpoždění na zákldě historických dat bude využita knihovna scikit-learn.

\end{itemize}

\subsubsection{Proces běhu aplikace}

Jak je již zmíněno aplikace bude využívat historická data, tedy bude nutné nechat aplikaci tato data nějakou dobu sbírat. Pro efektivní odhady by bylo vhodné mít uložené historické polohy vozidel alesponˇ z uplynulých několika týdnů.

\bigbreak

Avšak již v průběhu sběru dat může aplikace poskytovat základní službu a to vizualizování vozidel v mapě.
