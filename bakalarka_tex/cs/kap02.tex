\newcommand{\documentationAtt}[2] {#1 #2}

\chapter{Analýza zdroje dat}

V této kapitole je popsán zdroj real-timových dat o polohách vozidel využívané v této práci.

\section{Přístup k datům}

Na mnohých jednáníh s kolegy ze společnosti Operator ICT bylo řečeno, že využívané vozidla vysílají data o své poleze při různých událostech. Zejména pak při brždění, rozjezdu, ale také, pro účely této práce nejdůležitější, při vyhlášení zastávky, nebo jinak každých 20 sekund.

\bigbreak

Taková data pak přímo putují k provozavoateli systému na monitorování vozidel, kterým je společnost \gls{chaps} jakožto partner \gls{dpp}. Ten však tato data zpracovává a posílá ke zveřejnění na platformě Golemio. Bohužel při tomto procesu zpracování se vytratí informace o události v jáké byly data pořízeny. Tedy informace o příjezdu nebo odjezdu ze zastávky jsou zjistitelné pouze z \gls{gps} souřadnic.

\bigbreak

Po té co jsou tyto data přeneseny do společnosti Operátor ICT by měla být zveřejněna, nicméně data ve výše popsané podobě jsou poměrně chudá, proto je k nim přidáno více atributů. Z pohledu této práce je nejzajímavější informace o vzdálenosti, kterou vozidlo urazdilo od jeho výchozí zastávky. Dále jsou přidána data o jízdních řádech a zastávkách jejichž púvodcem je \gls{ropid}.

\subsection{Dokumentace}

Na úvod je nutné poznamenat, že datová platforma je stále ve vývoji a formát dat se může měnit. S tím mohou přicházet určité výpadky a problémy. K jednomu takovému výpadku došlu i při vývoji této práce, kdy po dobu 14 dnů platfomarma vůbec neodpovídala na dotazy nebo vracela prázdné datasety.

\bigbreak

Současně s využívaným datovým formátem, je nasazený pokročilý formát který obsahuje více informací a je přehledněji opraven. Nicméně při zahájení vývoje této práce nebyl k dispozici, proto jsou využívána data pouze ze starší verze.

\bigbreak

Oficiální dokumnetace datové platformy je poměrně zastaralá sama o sobě, tak že aktuální sada parametrů jí neodpovídá a neobsahuje žádné popisy dat. Proto vysvětlení jednotlivých atributů se zakládá na intuitivním pochopení nebo vyplynulo z jednání se správci platformy. V následujících kapitolách bude popsán formát dat, tak jak přichází od zdroje, a proto se může od oficiálně vystaené dokumentace lišit. A také budou popsány pouze atributy využívané v této práci nebo zajímavé pro její budoucí rozvoj.

TODO reference na dokumentaci

\bigbreak

Každá datová sada je exportována ve formátu \gls{geojson} pokud se jedná o geografická data, nebo jinak ve formátu \gls{json}. A přistupuje se k nim přes jednotné \gls{api} pomocí \gls{http} požadavku daného \gls{url} adresou a jeho hlavičkou.

TODO reference na specifikace geojson https://tools.ietf.org/html/rfc7946

\bigbreak

Ačkoli se dokumentace tváří tak, že data jsou exportována ve formátech \gls{json} nebo \gls{geojson}, většinou formát dat není přesně podle specifikace těchto formátů. Například může být uveden atribut \verb"wheelchair_accessible", který je typu \verb"bool" a je nastaven na hodnotu \verb"True", nicmně podle specifikace se tyto hodnoty píší s malým písmenem. Pro tuto práci to sice nepředstavuje komplikaci, protože tento atribut není potřeba, ale mohlo by se stát, že některé parsery \gls{json}u vyhodnotí řetězec jako nevalidní a skončí chybou.

\subsubsection{Pozice vozidel}

Jsou nejdůležijtější datovou sadou pro tuto práci. Jelikož se jedná o real-time data, data rychle zastarávají a je nutné je velmi často aktualizaovat.

\begin{itemize}
	\item \documentationAtt{coordinates}{aktuální \gls{gps} souřadnice vozidla}

	\item \documentationAtt{origin\_timestamp}{čas zachycení pozice vozidla, v časovém pásmu \gls{utc}}

	\item \documentationAtt{gtfs\_trip\_id}{unikátní identifik8tor tripu pro spárování s jízdním řádem}

	\item \documentationAtt{gtfs\_shape\_dist\_traveled}{vzdálenost vozidla uražená od začátku tripu v metrech}

	\item \documentationAtt{delay\_stop\_departure}{zpoždění zachycené při odjezdu z poslední projeté zastávky v sekundách}
\end{itemize}

\subsubsection{Jednotlivé tripy}

TODO jak se rekne trip cesky

Dále jsou k dispozici data o každém tripu. To je popis trasy vozidla, včetně zastávek a časů příjezdů a odjezdů do/z nich. Také může být vyžádáno k informacím o tripu připojit celý shape trasy, tj. lomená čára kopírující celo trasu daného tripu po povrchu Země.

\bigbreak

 Míra unikátnosti těchto tripů je předmětem dohadů a zřejmě jsou pod správou plánovačů \gls{mhd}, nicméně můžeme s určitou mírou spolehlivosti tvrdit, že každý trip se jede nejvýše jednou za den.

\begin{itemize}
	\item \documentationAtt{trip\_headsign}{nápis na čele vozidla, typicky cílová stanice}

	\item \documentationAtt{route\_id}{číslo linky}

	\item \documentationAtt{trip\_id}{unikátní identifikátor tripu pro spárování s real-time daty, pravděpodobně odpovídá atributu \verb"gtfs\_trip\_id"}
\end{itemize}

Navíc s každým tripem může být vyžádáno zaslání seznamu zastávek, kterýma projíždí. Po té se obdrží tento seznam s kompletními informacemi o zastávkách, tedy má stejnou informační hodnotu jako samostatný dotaz na zastávky. Navíc je každá zastávka doplněna o informace vázající se k danému tripu.

\subsubsection{Zastávky}

\begin{itemize}
	\item \documentationAtt{arrival\_time}{čas příjezdu spoje do zastávky}

	\item \documentationAtt{departure\_time}{čas odjezdu spoje do zastávky}

	\item \documentationAtt{shape\_dist\_traveled}{vzdálenost zastávky na trase od výchozího bodu daného tripuv metrech}

	\item \documentationAtt{stop\_id}{unikátní indetifikátor zastávky}

	\item \documentationAtt{coordinates}{\gls{gps} souřadnice zastávky, často \verb"None", je třeba využít atributy \verb"stop\_lat" a \verb"stop\_lon"}

	\item \documentationAtt{stop\_name}{název zastávky}
\end{itemize}



















TODO Vypozorováním zjištěno, že shape traveled je po celých 100 metrech.
